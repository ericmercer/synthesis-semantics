\documentclass[conference]{IEEEtran}
\IEEEoverridecommandlockouts
\usepackage{cite}
\usepackage{amsmath}
\usepackage{amsthm}
\usepackage{amsfonts}
\usepackage{url}
\usepackage{comment}
\usepackage{paralist}
\usepackage{graphicx}
\graphicspath{{./figs/}} 
\usepackage{listings}
\usepackage[usenames,dvipsnames,svgnames,table]{xcolor}
\usepackage{flushend}
\usepackage{hyperref}
% \usepackage[firstpage]{draftwatermark}

\definecolor{dkgreen}{rgb}{0,0.6,0}
\definecolor{mauve}{rgb}{0.58,0,0.82}
\definecolor{light-gray}{gray}{0.88}

\lstdefinestyle{myML}
{frame=none,
  basicstyle=\ttfamily,
  language=ML,
  aboveskip=3mm,
  belowskip=3mm,
  showstringspaces=false,
  columns=flexible,
  numbers=none,
  numberstyle=\tiny\color{gray},
  commentstyle=\color{dkgreen},
  stringstyle=\color{mauve},
  breaklines=false,
  breakatwhitespace=true,
  tabsize=2,
  linewidth=2\linewidth
}

\lstdefinestyle{agree}
{frame=none,
  basicstyle=\ttfamily,
  language=ML,
  aboveskip=3mm,
  belowskip=3mm,
  showstringspaces=false,
  columns=flexible,
  numbers=none,
  numberstyle=\tiny\color{gray},
  commentstyle=\color{dkgreen},
  stringstyle=\color{mauve},
  breaklines=false,
  breakatwhitespace=true,
  tabsize=2,
  linewidth=2\linewidth,
  morekeywords={eq, bool, guarantee, assume, true, false, pre, not, and, or, property, const}
}

\hyphenation{op-tical net-works semi-conduc-tor}

\newcommand{\konst}[1]{\ensuremath{\mathsf{#1}}}
\newcommand{\imp}{\Rightarrow}
\newcommand{\lval}{\ensuremath{\mathit{lval}}}
\newcommand{\set}[1]{\ensuremath{\{ {#1} \}}}
\newcommand{\kstar}[1]{\ensuremath{{#1}^{*}}}
\newcommand{\Lang}[1]{\ensuremath{{\mathcal L}({#1})}}
\newcommand{\LangTheta}[1]{\ensuremath{{\mathcal L}_{\theta}({#1})}}
\newcommand{\itelse}[3]{\mbox{$\mathtt{if}\ {#1}\ \mathtt{then}\ {#2}\ \mathtt{else}\ {#3}$}}

\newcommand{\figref}[1]{Fig.~\ref{#1}}

% Latex trickery for infix div operator, from stackexchange

\makeatletter
\newcommand*{\bdiv}{%
  \nonscript\mskip-\medmuskip\mkern5mu%
  \mathbin{\operator@font div}\penalty900\mkern5mu%
  \nonscript\mskip-\medmuskip
}
\makeatother

\newcommand{\ie}{\textit{i.e.}}
\newcommand{\eg}{\textit{e.g.}}
\newcommand{\etal}{\textit{et al.}}
\newcommand{\etc}{\textit{etc.}}
\newcommand{\adhoc}{\textit{ad hoc}}

\theoremstyle{plain}
\newtheorem{theorem}{Theorem}
\newtheorem{lemma}{Lemma}

\theoremstyle{definition}
\newtheorem{definition}{Definition}

\theoremstyle{remark}
\newtheorem{remark}{Remark}

\begin{document}
\title{
  Synthesizing Verified Cyber Assured Components
  \thanks{DISTRIBUTION STATEMENT A.  Approved for public release.}
}

% author names and affiliations
% use a multiple column layout for up to three different
% affiliations
\author{
\IEEEauthorblockN{Eric Mercer}
\IEEEauthorblockA{Department of Computer Science\\
Brigham Young University\\
Provo, Utah}
\and
\IEEEauthorblockN{
Konrad Slind, Isaac Amundson, Darren Cofer}
\IEEEauthorblockA{
Applied Research and Technology\\
Collins Aerospace\\
Minneapolis, Minnesota}
\and
\IEEEauthorblockN{
Junaid Babar, David Hardin}
\IEEEauthorblockA{
Applied Research and Technology\\
Collins Aerospace\\
Cedar Rapids, Iowa}
}
% conference papers do not typically use \thanks and this command
% is locked out in conference mode. If really needed, such as for
% the acknowledgment of grants, issue a \IEEEoverridecommandlockouts
% after \documentclass

\maketitle

\IEEEpeerreviewmaketitle

\section{Contiguity Types}
The formal specification of a component, and the synthesis of that specification, relies on \emph{contiguity types}~\cite{contiguity-types} to define the input and output data (cite contiguity). A contiguity type is a self-describing specification for messages. Its formalism has basis in formal languages. Similar to how a regular expression implies a set of words that form its language, so does a contiguity type specification imply a set of messages for its language where a message is a finite sequence of contiguous bytes (e.g., a string). 

What makes contiguity type specification more expressive than regular expressions is that it is self-describing meaning that the contents of the message itself may determine the rest of the message. An example is the \texttt{AutomationResponse} from the system in the previous section with its contiguity type specification.
{\small
\begin{verbatim}
  {TaskID : i64
   Length : u8
   Waypoints : Waypoint[Length]
  }
\end{verbatim}
}
\noindent The \texttt{Waypoints} array size depends on the value of \texttt{Length} so the actual number of bytes in the message depends on the contents of the message itself. 

The type specifications may also carry meta-information about the contents of the message.
{\small
\begin{verbatim}
  {Latitude : float
   lt-rng : Assert (-90 <= Latitude <= 90) 
   Longitude : float
   lng-rng : Assert (-180 <= Longitute <= 180)
   Altitude : float
   a-rng : Assert (10000 <= Altitude <= 15000)
  }
\end{verbatim}
}
\noindent Here the specification encodes the allowed ranges for each field of the waypoint. These assumptions restrict the resulting language to include only conforming messages and can be checked while constructing a message from a sequence of bytes. The notation $\LangTheta{\tau}$ denotes the language defined by the specification $\tau$ using the environment $\theta$ for expression evaluation.

Every contiguity type specification has a corresponding CakeML \emph{matcher} that when given a message string returns true or false if that message belongs to the language of the specification. If the message does belong to the language, an \emph{environment} is provided to access each part of the message. An environment, $\theta: \lval \mapsto \konst{string}$ binds \emph{L-values} to strings, where an L-value is an expression that can appear on the left hand side of an assignment (e.g., \texttt{AutomationRequest.Waypoints[0].Latitude}). 

The main result of contiguity types is the proof of the relationship between the language of the specification and the synthesized matcher from the specification that is summarized below.
\[
  \konst{match}\; s_1s_2 = \konst{SOME}(\theta, s_2)
  \imp \theta(\tau) \cdot s_2 = s_1s_2 \wedge s_1 \in \LangTheta{\tau}
\]
If there is a match on the substring $s_1$, then reconstituting the string from the resulting environment and concatenating it with $s_2$ yields the original string, and the matched string $s_1$ is in the language of the type specification. 

\section{Semantics}
\begin{figure}
  \[
    \begin{array}{rcl}
      \mathit{c}    & = & \konst{input}\ [(f : \tau)\ldots] \\
                    &   & \konst{output}\ [(f : \tau)\ldots] \\
                    &   & \konst{eq}\; [(f : \tau := \mathit{exp})\ldots] \\
                    &   & \konst{guarantee}\ [(\mathit{bexp})\ldots]\\\\
      
      \lval         & = & f \mid \lval \, [ \mathit{exp} ]
                          \mid \lval . f \\ \\

      f             & = & \mathit{varName} \\ \\

      \mathit{exp}  & = & \konst{Loc}\; \lval
                          \mid \konst{nLit}\; \konst{nat}
                          \mid \mathit{constname} \\
                    & | & \mathit{exp} + \mathit{exp}
                          \mid \mathit{exp} * \mathit{exp} \\
                    & | & (\mathit{exp}\ \rightarrow\ \mathit{exp}) \\
                    & | & (\konst{pre}\ \mathit{exp}) \\
                    & | & (\konst{ite}\ \mathit{bexp}\ \mathit{exp}\ \mathit{exp})\\
                    & | & \mathit{bexp} \\ \\
                          
      \mathit{bexp} & = & \konst{bLoc}\; \lval
                          \mid  \konst{bLit}\; \konst{bool}
                          \mid  \neg \mathit{bexp}
                          \mid  \mathit{bexp} \land \mathit{bexp} \\
                    & | & \mathit{exp} = \mathit{exp} 
                    \mid  \mathit{exp} < \mathit{exp}
\end{array}
\]
\caption{Syntax for high-assurance component specifications.}
\label{fig:syntax}
\end{figure}

The specification language for a high-assurance component is in \figref{fig:syntax}. A specification defines the inputs, outputs, local values, and guarantees for each output. A type $\tau$ is a contiguity type, and $(f : \tau)\ldots$ means zero or more repetition (e.g., Kleene star). An $\lval$ must eventually resolve to something that can be assigned. The expression language divides out Boolean expressions to simplify the semantics but is otherwise typical. The \konst{Loc} and \konst{bLoc} refer to the value of an $\lval$, while \konst{nLit} and \konst{bLit} indicate a literal. The language includes the initialization ($\rightarrow$), \konst{pre}, and if-then-else (\konst{ite}) operators.

Change the following paragraph and preceding paragraphs to define the specification as the core language with a well defined normal form where expressions are flat, lvals are unique etc. It takes care of all the normal semantic checks related to types, dependency order, etc. Anything in the core language has normal form and is perfect.

The semantics are only defined for \emph{well-formed} specifications. A specification is well-formed if and only if
\begin{compactenum}
\item Every $\lval$ is unique;
\item the \konst{eq} list is in dependency order and the expressions are acyclic;
\item the associated $\lval$ with each \konst{Loc} and \konst{bLoc} expression is a valid reference in the environment;
\item the associated literal with each \konst{nLit} and \konst{bLit} has the correct type;
\item \konst{pre} expressions do not refer past the beginning of the associated streams;
\item the expression list from \konst{guarantee} exactly corresponds in size and order to the list from \konst{output}; and
\item every expression in the list from \konst{guarantee} defines its corresponding output value under all input combinations.
\end{compactenum}
These checks are part of the synthesis but omitted to simplify the presentation.

An environment, $\theta: \lval \mapsto \konst{string}$ binds L-values to strings. 

The well-formed assumption enables the use of a single global environment for the semantics. The semantics are synchronous data-flow on a single clock defined over a sequences of environments where $\theta^i$ is the $i^\mathrm{th}$ environment in the stream. Expression evaluation is defined in the context of this environment stream as shown is \figref{fig:eval}. Here, $\konst{eval}\; i\; e$ carries with it the index of the environment to be used for the expression. $\Delta : \konst{string} \to \mathbb{N}$ binds constant names to numbers. Functions $\konst{toN}:\konst{string}\to\mathbb{N}$ and $\konst{toB}:\konst{string}\to\konst{bool}$ interpret byte sequences to numbers and booleans, respectively. 

\begin{figure*}
\[
\begin{array}{l}
\konst{eval}\; i\; e =
\mathtt{case}\; e\
 \left\{
 \begin{array}{lcl}
    \konst{Loc}\; \lval & \Rightarrow & \konst{toN}(\theta^i(\lval)) \\
    \konst{nLit}\; n & \Rightarrow & n  \\
    \mathit{constname} & \Rightarrow & \Delta(\mathit{constname})  \\
    e_1 + e_2 & \Rightarrow & \konst{eval}\; i \; e_1 + \konst{eval}\; i \; e_2  \\
    e_1 * e_2 & \Rightarrow & \konst{eval}\; i \; e_1 * \konst{eval}\; i \; e_2  \\
    e_1 \rightarrow e_2 & \Rightarrow &  \mathbf{if}\; i = 0\; \mathbf{then}\; \konst{eval}\; i \; e_1\; 
                                         \mathbf{else}\; \konst{eval}\; i \; e_2 \\
    (\konst{pre}\; e) & \Rightarrow &  \konst{eval}\; i-1 \; e
  \end{array}
 \right.
 \\ \\
\konst{evalB}\; i \; b =
\mathtt{case}\; b\
 \left\{
 \begin{array}{lcl}
    \konst{bLoc}\; \lval & \Rightarrow & \konst{toB}(\theta^i(\lval)) \\
    \konst{bLit}\; b & \Rightarrow & b \\
    \neg b & \Rightarrow & \neg(\konst{evalB} \; b)  \\
    b_1 \lor b_2 & \Rightarrow & \konst{evalB}\; i \;b_1 \lor \konst{evalB}\; i \;b_2   \\
    b_1 \land b_2 & \Rightarrow & \konst{evalB}\; i \;b_1 \land \konst{evalB}\; i \;b_2   \\
    e_1 = e_2 & \Rightarrow & \konst{eval} \;e_1 = \konst{eval}\; i \;e_2   \\
    e_1 < e_2 & \Rightarrow & \konst{eval} \;e_1 < \konst{eval}\; i \;e_2
  \end{array}
 \right.
\end{array}
\]
\caption{Expression evaluation in the context of a stream on environments.}
\label{fig:eval}
\end{figure*}

Each environment in the stream is initially partial meaning that it only contains mappings for the inputs. \emph{Stepping} the specification updates the current environment and checks the invariance of the guarantees. In other words, at the $i^\mathrm{th}$ step, $\theta^i$ is updated with the result of the sequential evaluation of the \konst{eq}-statements in the specification and then the guarantees are checked for invariance. 

The notation $(\lval \mapsto \mathit{slice}) \bullet \theta$ denotes the addition of binding $\lval \mapsto \mathit{slice}$ to $\theta$. Create an eval function for \konst{eq}-statement. Map it to the list of statements. Create a similar function for the guarantees with its map that fails if invariance does not hold. Define the language of the specification using the contiguity notation. Need to munge all the types into a single $\tau$, but the gist is the language is any finite stream possible that conform to the input specification are the result of the eq-statements, and are invariant. The set is prefix closed (trace theory).

\begin{comment}
There are 3 possible forms expected for a code guarantee, depending
on the output port type.
 
1. Event port. The expected form is
 
      event(port) = exp
 
    This indicates that port is an event port and it will be set (or not)
    according to the value of exp, which is boolean.
 
2. Data port. The expected form is
 
      port = exp
 
    This indicates that port is a data port and that the value of exp
    will be written to it.
 
3. Event data port. The expected form is
 
      if exp1 then
        event (port) and port = exp2
      else not (event port)
 
    This checks the condition exp1 to see whether an event on port will
    happen, and exp2 gives the output value if so. Note that any input
    event (or event data) port p occurring in exp2 must be guaranteed to
    have an event by event(-) checks in exp1.
 
In all of 1,2,3, the expressions should not mention any output ports,
i.e. the value to be sent out is determined by a computation over input
ports and state variables only.

Also related...

property Filter_policy = WELL_FORMED_MESSAGE(Input);
 
can be evaluated in a scenario where there is no input event. And so it will die, instead of returning false, even though the
later guarantee is only evaluated when there has indeed been an input event. Fix is to check for the event in the policy, as in
 
   property Filter_policy =
    if event (Input) then WELL_FORMED_MESSAGE(Input) else false;
 
   guarantee Filter_Output "The filter output shall be well-formed" :
     if event(Input) and Filter_policy then
        event(Output) and Output = Input
     else
       not event(Output);
 
OR, you could do away with the property and just have
 
   guarantee Filter_Output "The filter output shall be well-formed" :
    if event(Input) and WELL_FORMED_MESSAGE(Input) then
        event(Output) and Output = Input
     else
       not event(Output);
       
\end{comment}


%\section*{Acknowledgments}
%This work was sponsored in part by the Defense Advanced Research
%Projects Agency (DARPA).

% An example of a double column floating figure using two subfigures.
% (The subfig.sty package must be loaded for this to work.)
% The subfigure \label commands are set within each subfloat command,
% and the \label for the overall figure must come after \caption.
% \hfil is used as a separator to get equal spacing.
% Watch out that the combined width of all the subfigures on a 
% line do not exceed the text width or a line break will occur.
%
%\begin{figure*}[!t]
%\centering
%\subfloat[Case I]{\includegraphics[width=2.5in]{box}%
%\label{fig_first_case}}
%\hfil
%\subfloat[Case II]{\includegraphics[width=2.5in]{box}%
%\label{fig_second_case}}
%\caption{Simulation results for the network.}
%\label{fig_sim}
%\end{figure*}
%
% Note that often IEEE papers with subfigures do not employ subfigure
% captions (using the optional argument to \subfloat[]), but instead will
% reference/describe all of them (a), (b), etc., within the main caption.
% Be aware that for subfig.sty to generate the (a), (b), etc., subfigure
% labels, the optional argument to \subfloat must be present. If a
% subcaption is not desired, just leave its contents blank,
% e.g., \subfloat[].

% trigger a \newpage just before the given reference
% number - used to balance the columns on the last page
% adjust value as needed - may need to be readjusted if
% the document is modified later
%\IEEEtriggeratref{8}
% The "triggered" command can be changed if desired:
%\IEEEtriggercmd{\enlargethispage{-5in}}

% references section
\clearpage
\bibliographystyle{IEEEtran}
\bibliography{paper}

% \appendix
% \section{Contiguity Types}
% The formal specification of a component, and the synthesis of that specification, relies on \emph{contiguity types}~\cite{contiguity-types} to define the input and output data (cite contiguity). A contiguity type is a self-describing specification for messages. Its formalism has basis in formal languages. Similar to how a regular expression implies a set of words that form its language, so does a contiguity type specification imply a set of messages for its language where a message is a finite sequence of contiguous bytes (e.g., a string). 

What makes contiguity type specification more expressive than regular expressions is that it is self-describing meaning that the contents of the message itself may determine the rest of the message. An example is the \texttt{AutomationResponse} from the system in the previous section with its contiguity type specification.
{\small
\begin{verbatim}
  {TaskID : i64
   Length : u8
   Waypoints : Waypoint[Length]
  }
\end{verbatim}
}
\noindent The \texttt{Waypoints} array size depends on the value of \texttt{Length} so the actual number of bytes in the message depends on the contents of the message itself. 

The type specifications may also carry meta-information about the contents of the message.
{\small
\begin{verbatim}
  {Latitude : float
   lt-rng : Assert (-90 <= Latitude <= 90) 
   Longitude : float
   lng-rng : Assert (-180 <= Longitute <= 180)
   Altitude : float
   a-rng : Assert (10000 <= Altitude <= 15000)
  }
\end{verbatim}
}
\noindent Here the specification encodes the allowed ranges for each field of the waypoint. These assumptions restrict the resulting language to include only conforming messages and can be checked while constructing a message from a sequence of bytes. The notation $\LangTheta{\tau}$ denotes the language defined by the specification $\tau$ using the environment $\theta$ for expression evaluation.

Every contiguity type specification has a corresponding CakeML \emph{matcher} that when given a message string returns true or false if that message belongs to the language of the specification. If the message does belong to the language, an \emph{environment} is provided to access each part of the message. An environment, $\theta: \lval \mapsto \konst{string}$ binds \emph{L-values} to strings, where an L-value is an expression that can appear on the left hand side of an assignment (e.g., \texttt{AutomationRequest.Waypoints[0].Latitude}). 

The main result of contiguity types is the proof of the relationship between the language of the specification and the synthesized matcher from the specification that is summarized below.
\[
  \konst{match}\; s_1s_2 = \konst{SOME}(\theta, s_2)
  \imp \theta(\tau) \cdot s_2 = s_1s_2 \wedge s_1 \in \LangTheta{\tau}
\]
If there is a match on the substring $s_1$, then reconstituting the string from the resulting environment and concatenating it with $s_2$ yields the original string, and the matched string $s_1$ is in the language of the type specification.  

% \section{System Model and Semantics}
% \begin{figure}
  \[
    \begin{array}{rcl}
      \mathit{c}    & = & \konst{input}\ [(f : \tau)\ldots] \\
                    &   & \konst{output}\ [(f : \tau)\ldots] \\
                    &   & \konst{eq}\; [(f : \tau := \mathit{exp})\ldots] \\
                    &   & \konst{guarantee}\ [(\mathit{bexp})\ldots]\\\\
      
      \lval         & = & f \mid \lval \, [ \mathit{exp} ]
                          \mid \lval . f \\ \\

      f             & = & \mathit{varName} \\ \\

      \mathit{exp}  & = & \konst{Loc}\; \lval
                          \mid \konst{nLit}\; \konst{nat}
                          \mid \mathit{constname} \\
                    & | & \mathit{exp} + \mathit{exp}
                          \mid \mathit{exp} * \mathit{exp} \\
                    & | & (\mathit{exp}\ \rightarrow\ \mathit{exp}) \\
                    & | & (\konst{pre}\ \mathit{exp}) \\
                    & | & (\konst{ite}\ \mathit{bexp}\ \mathit{exp}\ \mathit{exp})\\
                    & | & \mathit{bexp} \\ \\
                          
      \mathit{bexp} & = & \konst{bLoc}\; \lval
                          \mid  \konst{bLit}\; \konst{bool}
                          \mid  \neg \mathit{bexp}
                          \mid  \mathit{bexp} \land \mathit{bexp} \\
                    & | & \mathit{exp} = \mathit{exp} 
                    \mid  \mathit{exp} < \mathit{exp}
\end{array}
\]
\caption{Syntax for high-assurance component specifications.}
\label{fig:syntax}
\end{figure}

The specification language for a high-assurance component is in \figref{fig:syntax}. A specification defines the inputs, outputs, local values, and guarantees for each output. A type $\tau$ is a contiguity type, and $(f : \tau)\ldots$ means zero or more repetition (e.g., Kleene star). An $\lval$ must eventually resolve to something that can be assigned. The expression language divides out Boolean expressions to simplify the semantics but is otherwise typical. The \konst{Loc} and \konst{bLoc} refer to the value of an $\lval$, while \konst{nLit} and \konst{bLit} indicate a literal. The language includes the initialization ($\rightarrow$), \konst{pre}, and if-then-else (\konst{ite}) operators.

Change the following paragraph and preceding paragraphs to define the specification as the core language with a well defined normal form where expressions are flat, lvals are unique etc. It takes care of all the normal semantic checks related to types, dependency order, etc. Anything in the core language has normal form and is perfect.

The semantics are only defined for \emph{well-formed} specifications. A specification is well-formed if and only if
\begin{compactenum}
\item Every $\lval$ is unique;
\item the \konst{eq} list is in dependency order and the expressions are acyclic;
\item the associated $\lval$ with each \konst{Loc} and \konst{bLoc} expression is a valid reference in the environment;
\item the associated literal with each \konst{nLit} and \konst{bLit} has the correct type;
\item \konst{pre} expressions do not refer past the beginning of the associated streams;
\item the expression list from \konst{guarantee} exactly corresponds in size and order to the list from \konst{output}; and
\item every expression in the list from \konst{guarantee} defines its corresponding output value under all input combinations.
\end{compactenum}
These checks are part of the synthesis but omitted to simplify the presentation.

An environment, $\theta: \lval \mapsto \konst{string}$ binds L-values to strings. 

The well-formed assumption enables the use of a single global environment for the semantics. The semantics are synchronous data-flow on a single clock defined over a sequences of environments where $\theta^i$ is the $i^\mathrm{th}$ environment in the stream. Expression evaluation is defined in the context of this environment stream as shown is \figref{fig:eval}. Here, $\konst{eval}\; i\; e$ carries with it the index of the environment to be used for the expression. $\Delta : \konst{string} \to \mathbb{N}$ binds constant names to numbers. Functions $\konst{toN}:\konst{string}\to\mathbb{N}$ and $\konst{toB}:\konst{string}\to\konst{bool}$ interpret byte sequences to numbers and booleans, respectively. 

\begin{figure*}
\[
\begin{array}{l}
\konst{eval}\; i\; e =
\mathtt{case}\; e\
 \left\{
 \begin{array}{lcl}
    \konst{Loc}\; \lval & \Rightarrow & \konst{toN}(\theta^i(\lval)) \\
    \konst{nLit}\; n & \Rightarrow & n  \\
    \mathit{constname} & \Rightarrow & \Delta(\mathit{constname})  \\
    e_1 + e_2 & \Rightarrow & \konst{eval}\; i \; e_1 + \konst{eval}\; i \; e_2  \\
    e_1 * e_2 & \Rightarrow & \konst{eval}\; i \; e_1 * \konst{eval}\; i \; e_2  \\
    e_1 \rightarrow e_2 & \Rightarrow &  \mathbf{if}\; i = 0\; \mathbf{then}\; \konst{eval}\; i \; e_1\; 
                                         \mathbf{else}\; \konst{eval}\; i \; e_2 \\
    (\konst{pre}\; e) & \Rightarrow &  \konst{eval}\; i-1 \; e
  \end{array}
 \right.
 \\ \\
\konst{evalB}\; i \; b =
\mathtt{case}\; b\
 \left\{
 \begin{array}{lcl}
    \konst{bLoc}\; \lval & \Rightarrow & \konst{toB}(\theta^i(\lval)) \\
    \konst{bLit}\; b & \Rightarrow & b \\
    \neg b & \Rightarrow & \neg(\konst{evalB} \; b)  \\
    b_1 \lor b_2 & \Rightarrow & \konst{evalB}\; i \;b_1 \lor \konst{evalB}\; i \;b_2   \\
    b_1 \land b_2 & \Rightarrow & \konst{evalB}\; i \;b_1 \land \konst{evalB}\; i \;b_2   \\
    e_1 = e_2 & \Rightarrow & \konst{eval} \;e_1 = \konst{eval}\; i \;e_2   \\
    e_1 < e_2 & \Rightarrow & \konst{eval} \;e_1 < \konst{eval}\; i \;e_2
  \end{array}
 \right.
\end{array}
\]
\caption{Expression evaluation in the context of a stream on environments.}
\label{fig:eval}
\end{figure*}

Each environment in the stream is initially partial meaning that it only contains mappings for the inputs. \emph{Stepping} the specification updates the current environment and checks the invariance of the guarantees. In other words, at the $i^\mathrm{th}$ step, $\theta^i$ is updated with the result of the sequential evaluation of the \konst{eq}-statements in the specification and then the guarantees are checked for invariance. 

The notation $(\lval \mapsto \mathit{slice}) \bullet \theta$ denotes the addition of binding $\lval \mapsto \mathit{slice}$ to $\theta$. Create an eval function for \konst{eq}-statement. Map it to the list of statements. Create a similar function for the guarantees with its map that fails if invariance does not hold. Define the language of the specification using the contiguity notation. Need to munge all the types into a single $\tau$, but the gist is the language is any finite stream possible that conform to the input specification are the result of the eq-statements, and are invariant. The set is prefix closed (trace theory).

\begin{comment}
There are 3 possible forms expected for a code guarantee, depending
on the output port type.
 
1. Event port. The expected form is
 
      event(port) = exp
 
    This indicates that port is an event port and it will be set (or not)
    according to the value of exp, which is boolean.
 
2. Data port. The expected form is
 
      port = exp
 
    This indicates that port is a data port and that the value of exp
    will be written to it.
 
3. Event data port. The expected form is
 
      if exp1 then
        event (port) and port = exp2
      else not (event port)
 
    This checks the condition exp1 to see whether an event on port will
    happen, and exp2 gives the output value if so. Note that any input
    event (or event data) port p occurring in exp2 must be guaranteed to
    have an event by event(-) checks in exp1.
 
In all of 1,2,3, the expressions should not mention any output ports,
i.e. the value to be sent out is determined by a computation over input
ports and state variables only.

Also related...

property Filter_policy = WELL_FORMED_MESSAGE(Input);
 
can be evaluated in a scenario where there is no input event. And so it will die, instead of returning false, even though the
later guarantee is only evaluated when there has indeed been an input event. Fix is to check for the event in the policy, as in
 
   property Filter_policy =
    if event (Input) then WELL_FORMED_MESSAGE(Input) else false;
 
   guarantee Filter_Output "The filter output shall be well-formed" :
     if event(Input) and Filter_policy then
        event(Output) and Output = Input
     else
       not event(Output);
 
OR, you could do away with the property and just have
 
   guarantee Filter_Output "The filter output shall be well-formed" :
    if event(Input) and WELL_FORMED_MESSAGE(Input) then
        event(Output) and Output = Input
     else
       not event(Output);
       
\end{comment}


\end{document}


